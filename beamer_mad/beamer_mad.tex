\documentclass[12pt]{beamer}
\usetheme{Madrid}
\usepackage[utf8]{inputenc}
\usepackage[english]{babel}
\usepackage{amsmath}
\usepackage{amsfonts}
\usepackage{amssymb}
\usepackage{graphicx}
\author{Manuel Menéndez}
\title{This is an example of Madrid Beamer presentation}
%\setbeamercovered{transparent} 
%\setbeamertemplate{navigation symbols}{} 
%\logo{} 
%\institute{} 
\date{22 feb 1973} 
%\subject{} 
\begin{document}

\begin{frame}
  \titlepage
\end{frame}

\begin{frame}
\tableofcontents
\end{frame}

%% ---------frame básico----------------------------- %%
\frame{
\frametitle{Frame title II} 

Por su parte: bla, bla
\vspace{\baselineskip}	
\begin{itemize}
	\item Item 1 bla bla de documentos no WYSIWYG.  
	\vspace{0.6\baselineskip}	
	\item Item 2.		
	\vspace{0.6\baselineskip}	
	\item Item 3.
	\vspace{0.6\baselineskip}	
	\item Es gratis.
	\vspace{0.6\baselineskip}	
	\item Es portable (independiente del sistema operativo). 
\end{itemize}
}

\begin{frame}{basic Frame}

\[\int_{-\infty}^{\infty} e^{-x^2} dx = \sqrt{\pi}\]

\end{frame}

\end{document}